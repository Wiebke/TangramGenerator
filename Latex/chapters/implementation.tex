\chapter{Implementation} \label{chapter:implementation}

The tangram generator and its associated interfaces for choosing one tangram out of a given number of presented ones and for solving a chosen tangram are implemented in JavaScript. JavaScript is a scripting language originally designed for adding interactivity to web pages by manipulating the structure and content of HTML-documents, but in recent years has also gained popularity in other domains like game development and server-side applications. Most modern browsers on both desktop and mobile devices include a JavaScript engine, which means that the user is not required to install additional frameworks for an application to execute properly. Other technologies for running client-side computations in a browser, like Java Applets, do not provide such widely spread support and have additionally experienced declining popularity due to security issues. In consequence, JavaScript is well suited for an application targeted to support various input paradigms on different devices \cite{mdn15}.

The Document Object Model (DOM) is an interface to HTML and XML documents. It allows accessing and changing the elements of a document and their properties as well as attaching event handlers to elements. Almost all changes in the interface of the tangram generator are realised with DOM manipulations. On startup, the web page contains structural elements for all parts of the interface that will be displayed during execution. This includes elements for each of the six tangrams, an area for playing the game and buttons for invoking processes not directly associated with a specific element. While some elements, like the buttons, are only hidden when first visiting the page, others have yet to be filled with content, like the elements displaying tangrams or the game play. When displayed, tangrams are drawn as Scalable Vector Graphics (SVG) \cite{w3c11}, exploiting the fact that SVG is XML-based. The elements of a SVG-element are therefore part of the DOM and can be treated like any other element. An alternative to using SVG as a drawing method is the HTML5 canvas element. Contrary to SVG, the canvas element is raster-based. After an element has been drawn it can not be updated in any way. Using SVG, moving a tan corresponds to updating the \verb|x| and \verb|y| attributes of the corresponding polygon. Achieving the same result with canvas on the other hand requires for the entire scene to be redrawn.
Another advantage of using SVG as the graphics rendering methods, is the possibility of attaching event handlers to SVG elements. Event handlers are functions that are executed in case a certain event happens. Typical events in the scope of web pages are events involving user interaction through mouse, keyboard and touch or browser actions. The tangram generator almost solely makes use of mouse and touch events in order to make the interface interactive on both desktop and mobile devices. Event handlers for clicking and dragging are attached to an element once it is added to the DOM and the translation, rotation and flipping of tans. 

JavaScript code is executed in a single thread and reacts asynchronously to events such as the ones described above. This implicates that heavier computations like the generation of a large number of tangrams block the simultaneous execution of any other code. Informing the user about the current state of the application during such computations is crucial to provide a satisfying user experience. Web workers \cite{w3c12} are a technology introduced to JavaScript to allow the execution of scripts in an additional thread in the background. In contrast to the main execution thread, workers cannot directly access the DOM or use methods and properties associated with the current window. They can however communicate with the main thread in the form of messages that can be handled in the same way as any other event. Thus, Web workers enable showing the progress of the generation process without having to repeatedly interrupt it. The web worker handling the entire generation process is started immediately after the webpage with the tangram generator is entered. Each time a tangram has been generated, the worker sends a message to the main thread, which then updates the progress bar accordingly. After the desired number of tangrams has been generated, the web worker finishes by sending the generated tangrams to the main thread. 

Which kind of messages can be exchanged between main thread and web workers is browser-dependent. Fortunately, all browsers are capable of sending \verb|String| messages between threads. JavaScript Object Notation (JSON) is a key-value based data-interchange format with which an object can easily be transformed into a sendable String, so that objects like the generated tangrams can be exchanged as well. An example for a JSON representation of tan can be seen below.

\begin{lstlisting}
{"tanType": 0,
 "anchor": {"x": {"coeffInt": 42, "coeffSqrt": -6}, 
            "y": {"coeffInt": 30, "coeffSqrt": -6}},
 "orientation": 3}
\end{lstlisting}

JSON objects are also used to send statistics about chosen tangrams and played games to a simple HTTP Server written in Node.js \cite{node}, a platform often referred to as server-side JavaScript. The server writes all JSON files it receives into a database. The database used here is MongoDB \cite{mongo}, which as a document-oriented database, functions very well with JSON objects.