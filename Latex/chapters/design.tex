\chapter{Design}\label{chapter:design}
This chapter deals with the algorithms specifically designed for the tangram generator. Before going into detail about these algorithms the overall structure of the application and its interface will be presented. 

Once the user enters the site, the application starts to pre-compute some values that are repeatedly used throughout the entire program, like direction vectors for tangrams, and then randomly generates tangrams using an algorithm described in section \ref{generate}. As this takes some time depending on the device and the number of tangrams generated, a progress bar indicates the current state of the generation progress. At all time some information about how to interact with the application is displayed below the main interface. In the very beginning, this also includes some information about tangrams in general. After the generation progress has finished the tangrams are sorted according to an interestingness measure and the six top ranked tangrams are displayed for the user to choose from. An option for generating new tangrams is also provided. Some candidates for interestingness measures are presented in section \ref{interesting}. If the user clicks on one of the displayed shapes, a bigger version of the tangram becomes visible and he or she can attempt to solve the puzzle by translating, rotating and flipping the puzzle pieces until they cover the whole given shape. At this point it is also possible to simply display the solution or receive a hint showing the position of one of the tans. 
In order to collect some statistics about which tangrams are preferred and how they are solved, some data is sent to a database and can then be used to derive new interesting measures. Each time the user chooses one of six tangrams, his choice along with a suitable representation of the presented tangrams is sent to a database. Statistics sent after solving a puzzle include the tangram itself as well as the time needed to solve the puzzle, the number of hints used and the number of actions performed to reach a solution.

The final user interface is shown in Figures \ref{choose} and \ref{game}.

\begin{figure}
\centering
    \includegraphics[width=0.9\textwidth]{figures/chose.png}
\caption{Interface showing six tangrams to the user to choose from}
\label{choose}
\end{figure}

\begin{figure}
\centering
    \includegraphics[width=0.9\textwidth]{figures/game.png}
  \caption{Interface allowing to solve a tangram}  
  \label{game}
\end{figure}

\section{Generation Process}
\label{generate}

Two approaches to randomly generate tangrams have been pursued. Both approaches have in common that they start out with generating an order for how the seven tans are placed and then position the first tangram. Subsequently new tans are placed by randomly choosing a vertex of one the already placed tans and choosing a vertex of the new tan. The new tan and the already placed pieces are then connected at those two points if the placement of the new tan does not violate the constraint that pieces cannot overlap. A collection of invalid placements is shown in Figure \ref{overlap}. If the placement fails, a new attempt of placing the tan is made. This process is continued until all 7 have been placed. 

\begin{figure}
\centering
    \includegraphics[width=0.7\textwidth]{figures/overlap.pdf}
  \caption{Invalid placements of the red tans when the respective blue one has already been present}  
  \label{overlap}
\end{figure}

Determining if a placement is valid is done in three steps. The first step uses the point-in-polygon algorithm mentioned before. All points of the newly placed tan as well as between one and four points inside the tan are tested for containment in other tans. The points inside of a tan are needed to correctly detect cases where one puzzle piece is entirely contained in another one. The direction vectors to these points are pre-calculated and include vectors to the center of each tan and depending on the size of a tan more vectors that are derived from partitioning tans in some way. Without modification this would introduce thirds and halfs into coordinates, which is why the direction vectors of tangrams have been scaled by six compared to the dimensions presented in chapter \ref{chapter:background}. This still does not correctly handle cases where a large tan is placed on top of a small tan such that it lies completely inside the newly placed tan. Thus the same test is conducted the other way around. The second step uses line segment intersection to determine if any of the segments of already placed tans intersect with the segments of a new tan.
Lastly, the bounding box of the tans including the newly placed tan is computed. The placement is then rejected if the the horizontal or vertical range is larger then a certain threshold. This step has been added in an attempt to avoid  sequences of tans that are only attached at one point and favour tangrams that are somewhat compact.

As a first naive approach an orientation for each tan is sampled at very beginning. The attachment point at the already placed tans is chosen first. Then all points of the new tan are considered as possible attachment points in random order. If none of the points can be used, a new attachment point of the placed tans is sampled. Unfortunately, the fixed orientations together with the range threshold imposes a very strong restriction on the generation process which can lead to configurations where some tans are still missing, but cannot be placed anymore. Furthermore, this generation process leads to mostly loosely connected tangrams. Therefore, the second approach chooses orientations dynamically. Additionally, this approach steers the computation towards tangrams where tans have many edges in common.

The second approach starts out by sampling an orientation for the first tan and then places it. The process then continues to find two attachment points as before, however here, the orientation is sampled based on a probability distribution that favours orientations where the edges of the new tan align with already places pieces. This probability distribution is computed by checking if any of the segments meeting in the attachment point align for any of the orientations and apply a larger weight to such orientations. If the tan cannot be placed with the sampled orientation, a new orientation is sampled from the already computed probability distribution, where the probability of the just attempted orientation is set to zero. If none of the orientation lead to a valid placement, a new attachment point is chosen.

\section{Interestingness Measures}
\label{interesting}
The following list shows all properties that are computed after a tangram is generated sorted into different categories. While some of these are appropriate to be used directly as interestingness measures, others might be more suitable for filtering generated tangrams in respect to a certain properties. Altogether the properties have been chosen in an attempt to depict concepts like difficulty, visual aesthetics and correspondence to real world objects. 
\begin{description}
  \item[Properties of the outline:] total number of vertices in the outline, number of vertices in the outline excluding holes, perimeter and number of hanging pieces \\
  	The first three properties are presumedly related to the compactness and therefore difficulty of tangrams. The number of hanging pieces is defined as the number of points where the outline touches itself and might capture the correspondence to real-world objects for a low non-zero result. 
  \item[Properties of holes:] number of holes, number of vertices involved in holes, total hole area and type of holes \\
  	These properties are potentially more useful for filtering on the supposition that tangrams with smaller holes that for example do not touch the outer outline are interesting. This is the reason for including the type of holes as a property. Holes can either touch the outer outline or not. For multiple holes, a mixed case can also occur.  
  \item[Properties of edges] longest edge, shortest edge, number of matches edges \\
  	Shapes that correspond to real world objects often have small spikes which should be captured by the computation of the shortest edge in the outline.
  \item[Properties of points] range in x and y, convex hull percentage, number of matched points \\
  	The first three properties again deal with the level of compactness. The convex hull percentage is measures by first calculating the convex hull of the outline of a shape and then determining the percentage of area covered by the original shape. Thus, shapes with a convex hull percentage close to one are almost convex. Those often do not reveal much information about a tangram pattern and are therefore assumed to be difficult to solve. 
  	The number of matched points refers to the number of pairs of points that lie at the same position. This measure is highest when multiple tans meet in one point. Ranking according to this number should result in somewhat star-like shapes. 
   \item[Symmetry:] A shape with symmetric features is assumed to be visually pleasing. Here, axis symmetry is considered. 
\end{description}

\section{Gameplay}

The central algorithm needed during gameplay as well as for displaying the tangrams is the computation of the outline of a collection of non-overlapping tans. Drawing tangrams as individual tans could lead to the user being able to infer some of the structure due to the existence of lines between tans, which cannot be removed by setting borders without giving away some unwanted hints elsewhere. Additionally, computing the outline of the tans placed by the user and checking it against the outline of a given shape provides a neat possibility to detect all correct solutions to a tangram instead of just the generated one. In order to efficiently check the equality of the given outline containing only coordinates with integer numbers against the inexact coordinates of puzzle pieces, a snapping mechanism has been introduced.

The first step of computing the outline is calculating a set of line segment candidates that could be involved in the outline. These are the line segments of each individual tan, however split up into multiple segments where points from other tans touch a segment. Line segments that occur twice in this computation lie somewhere inside the tangram and therefore can be removed completely. The remaining segments are then traversed in a somewhat similar way as point in the convex hull computation. The computation starts with the point with the lowest x, and lowest y coordinate if there are multiple points with the same x-value and initialises a last segment as a horizontal segment with the starting point as one endpoint. Then the segment in that point with the largest angle to the last segment is taken as a new last segment. This process is continued until the starting point is reached again and all points of the shape are contained by its outline. The second condition is needed to ensure that the process does not terminate to early in cases where the starting vertex should be revisited multiple times as the outline touches itself in that point. If the area of the computed outline is larger than the sum of areas of all puzzle pieces, the outline of the tangram contains holes that have yet to be calculated. However, holes are traversed in a very similar manner.
 \label{outline}
